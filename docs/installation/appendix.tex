\appendix
\clearpage
\chapter{Appendix}

\section{A. How to Query the Data Graph Summary ?}

The Data Graph Summary is modeled using the \emph{Dataset Analytics Vocabulary\footnote{\url{http://vocab.sindice.net/analytics}}} ontology. Below are some examples on how to use the Data Graph Summary.

\begin{itemize}
\item The following query returns the properties used with the class \url{http://schema.org/Person} in the dataset \texttt{bbc.co.uk}:

\lstset{language=SQL,basicstyle=\footnotesize}
\begin{lstlisting}
PREFIX an: <http://vocab.sindice.net/analytics>
PREFIX any23: <http://vocab.sindice.net/>
PREFIX d: <http://sindice.com/dataspace/default/domain/>

SELECT DISTINCT ?Predicate
FROM <http://sindice.com/analytics>
WHERE {
    ?node an:label ?l .
    ?l an:label <http://schema.org/Person> .
    
    ?edge an:source ?node ;
          an:label ?Predicate ;
          an:publishedIn d:bbc.co.uk .
}
LIMIT 20
\end{lstlisting}


\item The following query returns the most used classes in a dataset:

\lstset{language=SQL,basicstyle=\footnotesize}
\begin{lstlisting}
PREFIX an: <http://vocab.sindice.net/analytics>
PREFIX any23: <http://vocab.sindice.net/>

SELECT ?Dataset ?Class (SUM(?card) AS ?Cardinality)
FROM <http://sindice.com/analytics>
WHERE {
    ?node any23:dataset_uri ?Dataset ;
          an:cardinality ?card ;
          an:label ?l .
    ?l an:label ?Class .
}
GROUP BY ?Dataset ?Label
ORDER BY DESC(?Cardinality)
LIMIT 20
\end{lstlisting}
\end{itemize}

\section{B. How to setup SparQLed from scratch ?}

In this Section, we assume that you have no SPARQL endpoint and no RDF data available. We explain you how to setup a SparQLed instance from scratch.

\subsection{Example Datasets}

One may download one of the following datasets:
\begin{itemize}
    \item the European Nature Information System\footnote{\textcolor{blue}{\url{http://thedatahub.org/dataset/eunis}}} dataset (large, long setup):

    \begin{raggedleft}
    \framebox[\linewidth][l]{
        \parbox[r]{\linewidth}{\texttt{
            \$ wget http://eunis.eea.europa.eu/rdf/\{sites,habitats,\\\hspace{0.5cm}species,taxonomies\}.rdf.gz -{}-directory-prefix eunis
        }}
    }
    \end{raggedleft}

    \item the New York Times - Linked Open Data\footnote{\textcolor{blue}{\url{http://thedatahub.org/dataset/nytimes-linked-open-data}}} dataset (small, fast setup):

    \begin{raggedleft}
    \framebox[\linewidth][l]{
        \parbox[r]{\linewidth}{{\small\texttt{
            \$ wget http://data.nytimes.com/\{locations,organizations,\\\hspace{0.5cm}people,descriptors\}.rdf -{}-directory-prefix nytimes
        }}}
    }
    \end{raggedleft}
\end{itemize}

We need to load the downloaded RDF data into a SPARQL endpoint.
We create here a Sesame Native\footnote{\textcolor{blue}{\url{http://www.openrdf.org/doc/sesame2/users/ch07.html}}} endpoint:

\bigskip
\begin{raggedleft}
\framebox[\linewidth][l]{
    \parbox[r]{\linewidth}{\texttt{
        \$ java -cp target/sparql-summary-assembly.jar org.sindice.summary.Pipeline -{}-feed -{}-type NATIVE -{}-repository /path/to/\mbox{$<\!data\!>$}-native -{}-add /path/to/\mbox{$<\!data\!>$} -{}-addformat=RDF/XML
    }}
}
\end{raggedleft}
\hfill

The format of the ingested RDF data is in NTriples by default. A different format can be given through the option \texttt{\mbox{-{}-addformat}}.

\subsection{Compute the Data Graph Summary}

To compute the DGS over the downloaded data, we will execute SPARQL queries against the previous Native repository. This creates the file \texttt{/path/to/\mbox{$<\!data\!>$}-summary.nt.gz} containing the summary in the NTriple format.

\bigskip
\begin{raggedleft}
\framebox[\linewidth][l]{
    \parbox[r]{\linewidth}{\texttt{
        \$ java -cp target/sparql-summary-assembly.jar org.sindice.summary.Pipeline -{}-type NATIVE -{}-repository /path/to/\mbox{$<\!data\!>$}-native -{}-outputfile /path/to/\mbox{$<\!data\!>$}-summary.nt.gz
    }}
}
\end{raggedleft}
\hfill

We load the DGS triples into a new Sesame Native repository \texttt{/path/to/\mbox{$<\!data\!>$}-summary-native} under the named graph \url{http://sindice.com/analytics}. The reason to create a new repository is that Sesame locks the Native repository when in use, preventing another connection to be made.

\bigskip
\begin{raggedleft}
\framebox[\linewidth][l]{
    \parbox[r]{\linewidth}{\texttt{
        \$ java -cp target/sparql-summary-assembly.jar org.sindice.summary.Pipeline -{}-feed -{}-type NATIVE -{}-repository /path/to/\mbox{$<\!data\!>$}-summary-native -{}-domain http://sindice.com/analytics -{}-add /path/to/\mbox{$<\!data\!>$}-summary.nt.gz
    }}
}
\end{raggedleft}
\hfill

\subsection{Configure the Webapp}

Since we are now using a Native repository, the SparQLed configuration file needs to be updated:
\begin{itemize}
\item Edit the SparQLed XML configuration file:

\bigskip
\begin{raggedleft}
\framebox[\linewidth][l]{
    {\small \texttt{
        recommendation-servlet/src/main/resources/default-config.xml
    }}
}
\end{raggedleft}

\item Edit both endpoints, i.e., for the original RDF Graph and for the DGS, to point to the Native repositories:

\bigskip
\begin{raggedleft}
\framebox[\linewidth][l]{
    \parbox[r]{\linewidth}{\texttt{
        <backend>NATIVE</backend>\\\hspace{0.02cm}
        <backendArgs>/path/to/\mbox{$<\!data\!>$}-native</backendArgs>
    }}
}\marginnote{\textcolor{blue}{{\bfseries proxy} tag}}
\end{raggedleft}

\bigskip
\begin{raggedleft}
\framebox[\linewidth][l]{
    \parbox[r]{\linewidth}{{\small\texttt{
        <backend>NATIVE</backend>\\\hspace{0.02cm}
        <backendArgs>/path/to/\mbox{$<\!data\!>$}-summary-native</backendArgs>
    }}}
}\marginnote{\textcolor{blue}{{\bfseries recommender} tag}}
\end{raggedleft}
\end{itemize}

\subsection{Launching the SparQLed Webapp}

Before following the steps detailed in the Section~3, you have to grant tomcat6 (or, tomcat7) the ownership to the native repositories:

\bigskip
\begin{raggedleft}
\framebox[\linewidth][l]{
    \parbox[r]{\linewidth}{{\small\texttt{
        \$ sudo chown -R tomcat6:tomcat6 /path/to/\mbox{$<\!data\!>$}-native\\\hspace{0.02cm}
        \$ sudo chown -R tomcat6:tomcat6 /path/to/\mbox{$<\!data\!>$}-summary-native
    }}}
}
\end{raggedleft}

